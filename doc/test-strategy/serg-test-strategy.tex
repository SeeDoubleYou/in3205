\documentclass{article}
\usepackage{pslatex}

\newcommand{\todo}{\emph{To be done.}}

\title{SERG Test Strategy}
\author{Arie van Deursen}
\date{Version 0.2, 31 January 2010}
\begin{document}
\maketitle

\section{Introduction}
This document describes the test strategy adopted in software
development projects conducted under the responsibility of the
Software Engineering Research Group (SERG) of 
Delft University of Technology.

The document provides guidelines on how to conduct testing.
It is inspired by 
\begin{enumerate}
\item The concept of a strategy document as discussed by 
\cite[Ch.~20]{Pezze:2008}
\item The notion of a \emph{test catalog} as proposed by 
 \cite{Marick:1995} and discussed by \cite[Section~11.4]{Pezze:2008}.
\item The collection of \emph{test design patterns} provided by
 \cite{Binder:2000}.
\end{enumerate}

The present document is in use at the course \emph{Software Testing
and Quality Engineering}, and plays a role in research projects
conducted at SERG.

This is a working document which is continuously in progress.
During each project, the test strategy is evaluated, and if necessary
adjusted. In particular, whenever a fault slips through testing,
a critical \emph{root cause analysis} is conducted
\cite[Section 20.7]{Pezze:2008}, and if possible the testing strategy
is adjusted so that in future faults like these are more likely
to be captured.


\section{Test Process}

The basic test process involves the following steps:
\begin{itemize}
\item Requirements are explicitly documented.
\item A requirements specification immediately includes a (number of)
   test case specifications.
 \item Design decisions are explicitly documented.
 \item Requirements as well design documents are used to derive
  test case specifications before coding starts.
\item The catalog of test patterns (Section~\ref{Patterns}) is used
  to derive test cases where appropriate.
\item Test code is written as much as possible \emph{before} 
  the corresponding production code (in line with \cite{Beck:2003}
\item After implementing a feature, a coverage tool is used to analyze
  whether all code is indeed covered.
\end{itemize}


\paragraph{Assertions and Design by Contract}
Where possible, design decisions and coding assumptions are documented
by means of assertions. In particular:
\begin{itemize}
\item Class invariants are encoded in Boolean function \texttt{invariant}.
  Assertion checking the invariant are included at the start and at the
  end of every public method.
\item Method preconditions are encoded as assertions at the beginning
  of each method.
\item Method post conditions are encoded as assertions at the end of
  each method, or just before the return statement.
\item All assertions are side-effect free.
\item \begin{sloppypar}
  Exceptions are thrown in line with the principles of Design-by-Contract
  \cite{Meyer:1992}.
  \end{sloppypar}
\end{itemize}


%% \paragraph{Test Automation}

%% JUnit

%% Covertura, Emma

\newcounter{patterncnt}
\setcounter{patterncnt}{-1}

%\newcommand{\entry}[1]{\par\medskip\noindent\textbf{#1:}\ }
\newcommand{\entry}[1]{\item \emph{#1}:\ }
\newcommand{\pattern}[1]{
 \refstepcounter{patterncnt}
 \paragraph{Pattern \thepatterncnt: #1}
}
\newcommand{\model}{\entry{Model Under Analysis}}
\newcommand{\faults}{\entry{Fault Model}}
\newcommand{\adequacy}{\entry{Adequacy}}
\newcommand{\strategy}{\entry{Test Strategy}}

\section{A Catalog of Test Patterns} \label{Patterns}

\subsection{The Catalog Format}

\pattern{Template}
For each entry in the catalog, we present the following information:
\begin{itemize}
\model 
  The artefact / document that we inspect in order
  to arrive at a test case;
\faults 
  The sort of faults that we expect to find.
\strategy
  The testing steps that must be taken to have a good chance
  to find these fault. The steps are derived from the model under analysis.
\adequacy
  A way of indicating to what extent we are done when adopting
  this strategy. This is typically expressed as a percentage of elements
  in the model under analysis that is to be covered. 
  (see \cite[Chapter 9]{Pezze:2008}.
\end{itemize}

\subsection{Boundary Values}

\pattern{Boundary values}
\begin{itemize}
\model 
  Design or requirements documents
\faults 
  Boundaries often give rise to ``off-by-one'' errors
\strategy 
  For each boundary test one point that is exeactly on the
  boundary, one that is just off the boundary. 
  (see also the $1\times1$ strategy from \cite{Binder:2000},
  and the sample test catalog in \cite[Table 11.7]{Pezze:2008}
  inspired by \cite{Marick:1995}).
\adequacy
  A boundary is adequately checked if it is execued with one value
  on the boundary as well as one just off the boundary.
  Coverage is given by the number of adequately tested boundaries
  divided by the total number of boundaries occurring in the 
  specs / designs.
\end{itemize}

 
\subsection{Decision Logic}

\pattern{Basic Decision Logic} 
\label{basic-logic}
Requirements often involve simple if-then-else decision logic 
in one way or another, which is tested through this pattern.
\begin{itemize}
\model 
  Design or requirements document
\faults
  Either the coding of the condition could be wrong, 
  or the selected action taken in the
  ``then'' or the ``else'' branch could be the wrong one.
\strategy
  For each decision involving a ``then'' and an ``else'' part,
  two test cases are needed, one capturing the then, and one capturing
  the else part.
\adequacy
  Branch coverage, which also ensures that implicit (omitted) 
  else branches are covered as well
  (see \cite[Section 12.3]{Pezze:2008}).
\end{itemize}


\pattern{Complex Decision Logic}
More complex decision logic (compared to Pattern~\ref{basic-logic})
may involve conditions composed from a range of predicates and
Boolean operators. 
\begin{itemize}
\model
  The complex decision logic should be modeled through a decision table.
\faults
  Incorrect Boolean operators, incorrect bracketing, negation forgotten,
  cases forgotten, ...
\strategy
  Ensure each basic condition contained in the complex decision
   once yields true and once yields false
\begin{sloppypar}
\adequacy
  Modified Condition / Decision Coverage (MC/DC)
  (see \cite[Section 12.4, 14.3]{Pezze:2008}; called each-condition / all conditions
  by \cite{Binder:2000}).
\end{sloppypar}
\end{itemize}

\pattern{Exception Handling}
\begin{itemize}
\model
  Design documents including explicit 
  specifications of exceptions to be raised under specific circumstances.
\faults
  Incorrect raising and handling of exceptions.
\strategy
  Throwing exceptions will usually be done in if-then-else structures,
  and hence testing is covered by Pattern~\ref{basic-logic}.
  Therefore, testing exception handling should focus on ensuring that
  only specified exceptions can be propagated, and that other
  exceptions are caught and handled appropriately.
  See also \cite[Section 15.12]{Pezze:2008}.
  A challenge in testing exception handling is usually controllability:
  being able to trigger the event that causes the exception.
\adequacy
  The ``try'' as well as the ``catch'' block of each try-catch
  statement should be covered.
\end{itemize}

\subsection{File I/O}

\emph{Writing the File I/O patterns is left as an exercise.}



\subsection{Further Test Patterns}

Other test patterns that are used include:
\begin{itemize}
\item Category/Partition
\item Pairwise Combination Testing
\item Iteration
\item Polymorphism
\item State Machines
\item Decision Tables
\end{itemize}
Describing these is work in progress.


%% \begin{itemize}
%% \model 
%%   Requirements or design documents in which combinations of values
%%   from different domains determine the outcome of an operation
%% \faults
%%   Due to the potential combinatorial explosion of possible values,
%%   it is easy to oversee a case.
%% \strategy
%%   In the \emph{category-partition} approach the \emph{domains} are
%%   split into categories, out of which specific values are selected.
%%   Furthermore, for so-called \emph{single} or \emph{error}-values,
%%   combinations are not tested anymore.

%%   In the \emph{pairwise combination} approach
%% \adequacy
  
%% \end{itemize}



\bibliographystyle{apalike}
\bibliography{test}

\end{document}
